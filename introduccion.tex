%%%%%%%%%%%%%%%%%%%%

%porcentajes
El nuevo coronovirus SARS-CoV-2, que provoca la enfermedad del COVID-19, continua extendiendose por el planeta, a noviembre del 2020, los pa\'ises con la mayor proporci\'on de casos de COVID-19 en todo el mundo incluyen a Estados Unidos y la  India con un 20.25\% y 16.83\% respectivamente. Colombia representa un 2.25\% de los casos en todo el mundo. El virus SARS-CoV-2 es el s\'eptimo coronavirus conocido que infecta a los seres humanos; En las personas generalmente el coronoravirus causa s\'intomas leves, como tos o resfriado, pero el nuevo coronavirus ha provocado enfermedades respiratorias m\'as graves y muertes en todo el mundo \cite{28}. Actualmente no existen tratamientos espec\'ificos para COVID-19. Mientras se llevan a cabo estudios cl\'inicos para desarrollar nuevos medicamentos, las compa\~{n}\'ias farmac\'euticas y los grupos de investigaci\'on est\'an buscando con una variedad de experimentos los resultados de un tratamiento efectivo. 

Con el aumento de casos de Covid-19, sumado a la cuarentena obligatoria, la demanda de consultas m\'edicas de atenci\'on primaria presencial ha incrementado, con ello el uso de plataformas digitales con servicios de telesalud, por ello,  diferentes instituciones en Colombia han adoptado la modalidad virtual como los centros m\'edicos donde se ha evidenciado el impacto de las tecnolog\'ias de la informaci\'on por que ha complementado las destrezas de la presencialidad y con ello han logrado suplir las necesidades en este sector\cite{3}. Sin embargo, el pa\'is no ha alcanzado un desarrollo significativo en estas nuevas tecnolog\'ias debido a que no todas las personas y los grupos de trabajo disponen de las herramientas digitales necesarias para cumplir con las actividades requeridas. 

%REVISAR
La concepci\'on de proyectos en telemedicina es un desaf\'io con un buen futuro en Colombia, seg\'un datos de  las regulaciones por el ministerio de salud en la que se establecen disposiciones para la telesalud y par\'ametros para la pr\'actica de la telemedicina facilitando el acceso y la prestaci\'on de servicios de salud en cualquiera de sus fases (promoci\'on, prevenci\'on, diagn\'ostico, tratamiento y rehabilitaci\'on)\cite{3}. Dado a que la integraci\'on de sistemas de telemedicina han tenido un desarrollo significativo en diferentes pa\'ises donde se han evidenciado un uso exponencial de estos sistemas durante la crisis del COVID-19 y que por ello el sistema general de salud se ve obligado a aplicar tecnolog\'ias de comunicaci\'on en medicina dada la eventualidad, con ello se promete un futuro en Colombia por lo que es necesario desarrollar productos de apoyo para fortalecer directamente el \'area, pues la asistencia m\'edica virtual durante la pandemia es una forma segura y efectiva de evaluar y guiar el diagn\'ostico y el tratamiento de un paciente, minimizando el riesgo de transmisi\'on de la enfermedad.

 



%La presencia del COVID-19 ha acelerado el modelo de consulta médica de atención primaria presencial, aumentando el uso de plataformas digitales con servicios de telesalud.

%Las personas que han sufrido por este virus y han pasado por una unidad de cuidados intensivos o UCIS necesitan realizar terapia respiratoria que les permita recuperar su capacidad pulmonar. Los procedimientos actuales para realizar dicha terapia respiratoria se hacen con un dispositivos de ventilaci\'on mec\'anica. Un dispositivo conocido para la intervenci\'on de terapia respiratoria es el inspir\'ometro de incentivo, el cual,  para su uso necesita del acompa\~{n}amiento del especialista y donde la evualuaci\'on se hace manera cualitativa; teniendo en cuenta que el COVID-19 ha demostrado que puede transmitirse de una persona a otra con facilidad, pues cada persona infectada puede a su vez infectar a entre 2 y 3 personas. \cite{29} Raz\'on por la cual, aparece la necesidad de un producto de uso remoto de apoyo para la terapia del paciente. Para el criterio y el dise\~{n}o se tiene en cuenta la opini\'on del especialista y aspectos como conciencia, colaboraci\'on y entendimiento por parte del paciente para el tratamiento efectivo de un paciente.

%Por lo anterior se plantea dise\~{n}ar e implementar un producto que de soluci\'on a la necesidad relacionada con la rehabilitaci\'on pulmonar post COVID-19, para ello se hace el an\'alisis respectivo de los requerimientos iniciales que necesita este producto utilizando una metodolog\'ia llamada, teor\'ia de soluci\'on de problemas inventivos TRIZ, la cual nos permite ampliar el an\'alisis a trav\'es de un trabajo interdiciplinar y es \'util para el dise\~{n}o de nuevos productos como un abanico de \'areas incluido el campo de la salud, de esa manera para la determinaci\'on de dichos requerimientos se designa un conjunto de fases que se plantean de la siguiente manera \cite{3}. En la primera fase se identifica el contexto de una necesidad,  en la segunda fase se describe cual es el problema que se va abordar a partir de la definici\'on de las necesidades planteadas por los usuarios, ya en la tercera fase se hace la descripci\'on de un sistema t\'ecnico compuesto por elementos que en conjunto llevan a cabo una funci\'on principal, es decir un sistema producto que toma la energ\'ia del exterior la transforma para realizar cierta actividad, en la cuarta fase denominada espacial y temporal se utiliza una t\'ecnica de nueve ventanas de TRIZ que nos permite analizar el problema en un contexto mas amplio con el fin de identificar diversas ideas para determinar los requerimientos, en la quinta fase con la ayuda de fuentes primarias y secundarias se determina cuales serian los atributos ideales que deber\'ia tener el sistema t\'ecnico en el futuro, en esta fase se hace crucial la participaci\'on del usuario, luego en la sexta fase se analiza por que lo que se plantea como ideal en el futuro no puede ser llevado a cabo con el sistema actual, de esta manera se genera una contradicci\'on, lo cual surge al pretender mejorar algo con una soluci\'on preliminar pero se desmejora otro atributo, finalmente utilizando elementos de la teor\'ia TRIZ que permite encontrar soluci\'on a las contradicciones determinamos los requerimientos. \cite{3}


Este proyecto es planteado en conjunto con un equipo de trabajo que esta conformado por personas de diferentes \'areas de la academia en la Pontificia Universidad Javeriana Cali, como la ingener\'ia, las ciencias de la salud, el dise\~{n}o industrial y el \'area de ingenier\'ia de software donde se enfocara el funcionamiento del producto, de igual forma se tiene la comunicaci\'on con el usuario que son el especialista y el paciente que hacen parte importante para la definici\'on de requerimientos. En particular, en este proyecto de grado se desarrollar\'a el software que se podr\'a integrar a un inspir\'ometro electr\'onico lo cual, en conjunto hacen parte de un proyecto de telemedicina. 

%Como parte del proceso inicial se hace la descripci\'on del problema a continuaci\'on,  generalmente en un panorama amplio, el problema se debe a la alta tasa de contagio por COVID-19 y los consecuencias que esto puede traer, de esa manera el paciente con coronavirus es sometido a confinamiento y distanciamiento social requiriendo al mismo tiempo atenci\'on m\'edica, por lo tanto la necesidad de dise\~{n}ar un producto que pueda ser usado de manera aut\'onoma y remota para el respectivo an\'alisis cl\'inico. Luego, se describen los conceptos fundamentales para entender el planteamiento y la soluci\'on parcial del problema.
%tomar el producto o sistema actual que soluciona de manera parcial el problema y se modela como un sistema t\'ecnico, este producto es el incentivo respiratorio inspir\'ometro y al ser un producto de apoyo, el usuario hace parte del sistema, con ello surge una nueva necesidad, que es la de desarrollar un software que permita la comunicaci\'on directa entre el sistema y el personal de salud.

%Para el desarrollo de este proyecto se cuenta con la t\'ecnica de nueve ventanas de TRIZ que nos permite hacer un an\'alisis del contexto del producto de manera amplio y a la vez reducido, es un an\'alisis en tiempo y espacio donde se evalua que el sistema que se va a plantear cumple con necesidades espec\'ificas que no est\'an cubiertas en la actualidad.  Al ser un producto software que se proyecta desarrollar se utilizar\'a la tecnolog\'ia de gamificaci\'on que dadas las revisiones de la literatura, esta tecnolog\'ia es incluida como parte de un modelo de tratamiento actual por los especialistas de la salud debido a que motiva a un paciente a realizar un procedimiento preescrito. 




%Es importante identificar los impactos que estas energ\'ias alternativas pueden ocasionar en las redes y as\'i prever posibles problemas que ya est\'an evidenciando en los sistemas de generaci\'on desarrollados. Por lo cual, se propone como caso de estudio la evaluaci\'on de los potenciales impactos en estabilidad de tensi\'on de estado estable y calidad de energ\'ia en contenido arm\'onico de tensi\'on y corriente en la micro red que presta servicio al campus de la Pontificia Universidad Javeriana Cali causados por la integraci\'on de un sistema de generaci\'on solar FV de hasta 400Kwp sin almacenamiento de energ\'ia.\\

%https://www.rtve.es/noticias/20201113/mapa-mundial-del-coronavirus/1998143.shtml
%\https://www.statista.com/statistics/1111696/covid19-cases-percentage-by-country/






