%%%%%%%%%%%%%%%%
% ABSTRACT
%%%%%%%%%%%%%%%%
%Resumen de la propuesta. (1/2 página). 

El aumento exponencial de usos de sistemas de telemedicina ha generado importantes iniciativas para la implementaci\'on de nuevos productos de apoyo que complementen en las actividades de seguimiento a pacientes que lo requieren; el manejo de las tecnolog\'ias de la informaci\'on en el sector de la salud permite reforzar los servicios que actualmente se encuentran de mandados por la crisis del COVID-19. El prop\'osito de este proyecto es desarrollar un producto de apoyo que pueda ser integrado a un dispositivo inspir\'ometro y que pueda ser manejado de manera remota por un paciente a quien se le ha prescrito un tratamiento respiratorio. Se plantea el uso de metodolog\'ias y t\'ecnicas innovadores para la determinaci\'on de los requerimientos y el desarrollo de software que se explicaran a lo largo del documento.

%La desmesurada evolución de la arquitectura de TI ha generado cambios importantes en el ámbito de la información; es así como los servicios de TI actuales se prestan basados en infraestructuras de servidores virtualizados a partir de los cuales se generan grandes ahorros en diferentes aspectos. El propósito del proyecto es plantear una solución de alta disponibilidad en contenedores Docker para la Pontificia Universidad Javeriana Cali, que permita alojar las diferentes aplicaciones con las que cuenta, disminuir consumo de recursos, estandarizar servidores, entre otras soluciones. Se plantea el uso de diferentes tecnologías que serán evaluadas a lo largo del documento, así como un trabajo detallado en verificación del estado actual y alternativas, para una posterior etapa detallada de diseño, otra de validación mediante un prototipo y una última de evaluación de resultados y desempeño.
%{\bf Palabras Clave}: Lista de palabras clave. 